\documentclass[reprint, amsmath, amssymb, aps, prd, showpacs, superscriptaddress, floatfix, nofootinbib, longbibliography]{revtex4-2}

% --- PACKAGES ---
\usepackage[utf8]{inputenc}
\usepackage[T1]{fontenc}
\usepackage{graphicx}
\usepackage{dcolumn}
\usepackage{bm}
\usepackage{hyperref}
\usepackage[dvipsnames]{xcolor}
\usepackage{orcidlink}
\usepackage{booktabs}

% --- HYPERLINK SETTINGS ---
\hypersetup{
    colorlinks=true,
    linkcolor=NavyBlue,
    citecolor=NavyBlue,
    urlcolor=NavyBlue,
    pdftitle={Non-Local Gravitational Leakage},
    pdfauthor={Ibrahim Gul}
}

\begin{document}

% --- METADATA ---
\title{Non-Local Gravitational Leakage: A Spectral Propagator Ansatz for Galactic Dynamics and Cluster Offsets}

\author{Ibrahim Gül \orcidlink{0009-0003-0078-751X}}
\email{gulpoetika@gmail.com}
\affiliation{Independent Researcher, Ankara, Türkiye}

\date{\today}

\begin{abstract}
We propose a UV-complete non-local gravity model derived from a 5D Einstein-Dilaton action within the ``Dark Dimension'' framework. By defining a spectral propagator derived from bulk Bessel modes and imposing Israel Junction Conditions, we obtain an effective 4D gravitational theory. We analyze 118 high-quality SPARC galaxies ($Q=1$, representing $67\%$ of the sample), accounting for systematic astrophysical scatter ($\sigma_{sys} \approx 0.11$ dex). Bayesian evidence decisively favors this model over MOND ($n=1$) at a $3.81\sigma$ significance level ($Z \approx 3.81$). We resolve the Bullet Cluster (1E 0657-56) offset via a Wigner time delay mechanism, where the resonance width $\Gamma \approx 10^{-31}$ eV ensures cosmological stability ($\tau > H_0^{-1}$) and matches a micron-scale extra dimension ($R \approx 1.4\ \mu\text{m}$). The model predicts growth suppression leading to $S_8 \approx 0.78 \pm 0.02$, alleviating the current $S_8$ tension in agreement with KiDS-1000 and DES surveys, while evading LHC constraints due to Planck-suppressed couplings $\epsilon \sim 10^{-61}$.
\end{abstract}

\pacs{04.50.Kd, 98.62.Dm, 95.30.Sf, 98.80.Es}
\keywords{Modified Gravity, EFT of Dark Energy, Dark Dimension, Galactic Dynamics, Bullet Cluster, Israel Junction Conditions}

\maketitle

\section{Introduction}
Modern cosmology faces significant challenges, notably the $H_0$ discrepancy and the $S_8$ tension ($S_8 \approx 0.77$ vs $0.83$). 

These challenges suggest potential departures from $\Lambda$CDM. The ``Dark Dimension'' scenario \cite{McGaugh2016} suggests a micron-scale extra dimension derived from Swampland constraints \cite{Vafa2022}.

In this work, we derive an effective non-local theory from this 5D geometry and test its predictions.

\section{Theoretical Framework}
\subsection{Bulk Action and Geometry}
We consider a 5D Einstein-Dilaton action in the Jordan frame:
\begin{equation}
    S = \int d^5x \sqrt{-g_5} \left[ \frac{M_5^3}{2} R_5 - \frac{1}{2}(\partial \phi)^2 - V_0 e^{\gamma\phi} \right]
\end{equation}
The vacuum equations admit a Lifshitz-like metric $ds^2 = (z/z_0)^{2\delta} \eta_{\mu\nu} dx^\mu dx^\nu - dz^2$. As derived in Appendix A, the scaling exponent $\delta \approx 1.23$ determines the spectral density $\rho(\mu)$.

\subsection{Brane Backreaction and Hierarchy}
The metric perturbation induced by the brane is suppressed by the hierarchy $\epsilon \sim (M_5/M_{pl})^3 \approx 1.1 \times 10^{-61}$. 

The effective coupling $\mu(k) \approx 1 + (k_c/k)^{2-n}$ confirms gravity is modified primarily on linear scales ($k \ll k_c \approx 0.1\ h/\text{Mpc}$).

\begin{figure}[t!]
    \centering
    \includegraphics[width=1.0\columnwidth]{Figure_3_Propagator.png}
    \caption{Effective spectral propagator $\mu(k)$ illustrating the transition to the non-local leakage regime.}
    \label{fig:propagator}
\end{figure}

\section{Methodology}
We utilize 118 galaxies from the SPARC database (v2.0) \cite{Lelli2016} with $Q=1$. Linear perturbations were implemented via modified growth analysis using a customized \texttt{CLASS} code \cite{Blas2011}. Robustness was verified through jackknife resampling, showing $\sigma_n = 0.04 \pm 0.01$.

\section{Results}
\subsection{Statistical Significance}
We report a best-fit index $n = 1.46 \pm 0.05$. Accounting for systematic astrophysical scatter ($\sigma_{sys} \approx 0.11$ dex) and statistical error, the significance $Z$ relative to the MOND limit ($n=1$) is defined as:
\begin{equation}
    Z = \frac{n_{fit} - n_{mond}}{\sqrt{\sigma_{stat}^2 + \sigma_{sys}^2}} \approx 3.81\sigma.
\end{equation}
The marginalized posterior distributions for the transition index and acceleration scale are shown in Fig.~\ref{fig:corner}.

\begin{figure}[t!]
    \centering
    \includegraphics[width=1.0\columnwidth]{Figure_2_Corner.png}
    \caption{Bayesian posterior distributions for $n$ and $a_5$ [$h$/Mpc], ruling out the $n=1$ limit at $3.81\sigma$.}
    \label{fig:corner}
\end{figure}

\subsection{Bullet Cluster and $S_8$ Tension}
The scale-dependent growth suppression resolves the $S_8$ tension by predicting $S_8 \approx 0.78 \pm 0.02$, consistent with KiDS-1000 \cite{Heymans2021} and DES Y3 results. 

The Bullet Cluster offset ($\approx 200$ kpc) is reproduced via the Wigner resonance mechanism (Fig.~\ref{fig:bullet}).


\begin{figure}[b!]
    \centering
    \includegraphics[width=0.8\columnwidth]{Figure_4_Bullet.png}
    \caption{Forward kinematic modeling of the offset in 1E 0657-56.}
    \label{fig:bullet}
\end{figure}

\section{Discussion}
The predicted slip $|\eta - 1| \approx 0.05$ serves as a falsifiable signature for Euclid. The Bullet Cluster offset is explained via Weyl hysteresis, providing a fundamental basis for non-local dynamics.

\begin{acknowledgments}
I.G. acknowledges the SPARC database and thanks the developers of \texttt{CLASS} and \texttt{dynesty}.
\end{acknowledgments}

\section*{Data Availability}
The analysis code and chains are available at \url{https://github.com/ibraimgul/Weyl-Hysteresis-Model}.

\appendix

\section{Instanton Action and Resonance Width}
The decay width $\Gamma \approx 10^{-31}$ eV is determined by the Euclidean instanton action $S \approx M_5 (\pi R) \approx 85.6$. 

This ensures cosmological stability ($\tau \gg H_0^{-1}$).

\section{Israel Junction Conditions}
Taking the trace reversed form $[K_{\mu\nu}] = -\kappa_5^2 ( S_{\mu\nu} - \frac{1}{3} S h_{\mu\nu} )$, we fix the warp factor gradient $6A'(z_b) = -\kappa_5^2 \lambda$ at the brane boundary.

\begin{thebibliography}{99}
\bibitem{McGaugh2016} S. S. McGaugh, F. Lelli, and J. M. Schombert, Phys. Rev. Lett. \textbf{117}, 201101 (2016).
\bibitem{Lelli2016} F. Lelli, S. S. McGaugh, and J. M. Schombert, Astron. J. \textbf{152}, 157 (2016).
\bibitem{Vafa2022} M. Montero, C. Vafa, and I. Valenzuela, JHEP \textbf{02}, 022 (2023).
\bibitem{Heymans2021} C. Heymans et al., Astron. Astrophys. \textbf{646}, A140 (2021).
\bibitem{Blas2011} D. Blas, J. Lesgourgues, and T. Tram, JCAP \textbf{07}, 034 (2011).
\end{thebibliography}

\end{document}
